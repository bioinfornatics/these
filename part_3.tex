
\begin{refsegment}
\chapter*{Perspectives pour GROOLS}
\addcontentsline{toc}{chapter}{Perspectives pour GROOLS}

Ces trois années de recherche m'ont permis de mettre au point un raisonnement pour l'annotation fonctionnelle de génomes bactériens. La représentation de ce domaine de la biologie, dans un système expert s'est montré ardue, mais essentiel pour améliorer le processus de curation. En effet, j'ai démontré que la logique paracohérente à trois ou quatre valeurs de vérités ne permettait pas de répondre à la problématique. J'ai donc conçu une représentation de l'information sous forme d'ensemble de valeurs de vérités permettant d'obtenir une meilleure description des observations attachées aux différents concepts. La structuration hiérarchiques des connaissances \textit{a priori} (théories) mis en place permet d'intégrer toutes ressources dont les concepts sont reliées par des liens de compositions et/ou d'équivalences. Cette organisation permet notamment d'utiliser les connaissances sur les réseaux métaboliques  de \textit{Genome Properties} et d'\textit{UniPathway}. Grâce à cette représentation et aux ensembles de valeurs de vérités le raisonneur est capable de suggérer des chemins de réactions alors que les données sont incomplètes. Le biologiste peut rapidement évaluer le niveau de complétion de l'annotation génomique. Le système permet au biologiste de repérer rapidement les réactions manquantes et ambiguë. L'outil \texttt{GROOLS} est un train d'union entre la bio-informatique et la biologie. Il est capable d'utiliser intelligemment les connaissances du domaine et les résultats expérimentaux effectués en laboratoire afin de valider les théories.

Ce puissant outil ouvre de nouvelles pistes de recherches pour l'annotation des génomes, mais pas seulement. L'outil \texttt{GROOLS} peut également être utilisé pour d'autres domaines disposant de connaissances hiérarchiques.

Le système de représentation des connaissances peut être étendues selon deux axes. Le premier axe consiste a intégrer l'information taxonomique dans le graphe de connaissances. Pour l'étude des 14 génomes bactériens, l'observation et la connaissance taxonomique n'est pas dissocier, des observations type expectations ont directement été relié aux voies de biosynthèses des acides aminés. Cette amélioration permet de mettre en relation les notions de prototrophies et auxotrophies avec les voies métaboliques. Pour ce faire, il est nécessaire d'introduire un nouveau type de relation afin mettre en lien deux théories antagonistes.  Par exemple, pour un organisme aérobie strict on ne s'attends pas a retrouver la voie anaérobie. Ainsi une expectation ($\{t\}$) sur la connaissance \textit{a priori} aérobie strict a pour conséquence logique d'inférer l'ensemble $\{f\}$ sur le concept anaérobie. La présence d'un concept interdit la présence d'un autre ("avoid"). Le second axe à pour objectif d'identifier des gènes candidats pour les réactions marquées comme manquantes. Pour cela on peut représenter les relations \gls{GPR} dans le graphe de connaissances.

Dans l'objectif de compléter l'annotation de génome bactérien, les réactions validées présentes et absentes dans une voie métabolique d'un organisme peuvent être utilisées comme un profile. Ces profiles sont des chemins de réactions possibles utilisés par le vivant. A l'aide d'un système de recommandation, ces profiles peuvent être employés pour compléter les annotations de génomes lorsque deux organismes ont des profils proches. Pour éviter une sur-prédiction, les réactions ou les profils (selon la méthode) peuvent être étiqueter en trois familles: (i) présent partout ("core") , (ii) modérément présent ("shell"), (iii) rarement présent ("cloud").


En ce qui concerne l'interface utilisateur, le raisonnement réactif mis en place permet de réagir à l'ajout et à la suppression  d'observations . Cette fonctionnalité peut être pleinement exploitée à travers une application ( web ou "desktop") pour permettre au biologiste d'interagir directement avec le graphe de connaissances. Il peut ainsi tester des théories et observer directement les conséquences logiques.

\subbibliography
\end{refsegment}