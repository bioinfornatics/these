\begin{refsegment}
\chapter*{Perspectives}
\addcontentsline{toc}{chapter}{\color{MidnightBlue}\large\textsc\bfseries{Perspectives}}

Ces trois années de recherche m'ont permis de mettre au point un système expert, nommé \texttt{\gls{GROOLS}}, pour l'annotation fonctionnelle de génomes au travers de processus biologiques comme les voies métaboliques. Ces travaux ont tout d'abord consisté à choisir une représentation des connaissances et un raisonnement adaptés à la structure hiérarchique des données biologiques. En effet, une utilisation simple de la logique paracohérente à trois ou quatre valeurs de vérité ne permet pas de répondre complétement à la problématique. J'ai donc utilisé une représentation de l'information sous la forme d'ensembles de valeurs de vérité permettant d'obtenir une meilleure description des observations attachées aux différents concepts. La structuration hiérarchique des connaissances \textit{a priori} (théories) mise en place permet d'intégrer toutes ressources dont les concepts sont reliés par des liens de composition et/ou de généralisation. Cette représentation permet notamment d'utiliser les connaissances sur les réseaux métaboliques  de \textit{Genome Properties} et d'\textit{UniPathway} qui peuvent être étendues sans grande difficulté à d'autres ressources comme MetaCyc. L'utilisation des ensembles de valeurs de vérité permet au raisonneur de suggérer des variants de voies métaboliques avec des données biologiques qui peuvent être incomplètes et contradictoires. Le biologiste peut ainsi rapidement évaluer le niveau de complétion de l'annotation génomique. De plus, le système permet de repérer rapidement les réactions manquantes et les observations ambiguës. L'outil \texttt{GROOLS} est un "trait d'union" entre la bio-informatique et la biologie. Il est capable d'utiliser intelligemment les connaissances du domaine, les prédictions bio-informatiques et les résultats expérimentaux effectués en laboratoire afin de valider les théories.

L'outil a été évalué au travers de l'analyse de 14 génomes dont les organismes disposaient de données de phénotypes de croissance. La qualité originale du raisonnement de \texttt{\gls{GROOLS}} en fait un outil d'intérêt pour l'annotation des génomes mais l'outil \texttt{\gls{GROOLS}} pourrait également être utilisé dans d'autres domaines disposant de connaissances hiérarchiques.

Le système de représentation des connaissances peut être étendu selon deux axes. Le premier axe consiste à intégrer l'information de phénotype dans le graphe de connaissances. Dans l'étude précédemment menée sur 14 génomes bactériens, une observation de phénotype de croissance était directement associée à la voie dégradation correspondante. L'introduction de concepts de type phénotype dans le graphe de connaissances permettrait de mettre en relation un phénotype avec de multiples voies métaboliques. De plus, l'ajout d'un nouveau type de relation (i.e. "avoid" pour interdit) autoriserait la formulation de phénotypes plus complexes. Par exemple, pour une bactérie aérobie stricte, on ne s'attend pas à retrouver de voies anaérobies. Ainsi une expectation ($\{t\}$) sur le concept aérobie strict a pour conséquence logique d'inférer l'ensemble $\{f\}$ sur le concept anaérobie par une relation "avoid", en plus de la valeur $\{t\}$ sur le concept aérobie. Un second objectif serait d'associer des concepts "prédicteurs" à chaque unité fonctionnelle (e.g. réactions) à l'image de Genome Properties. Un attribut additionnel sur ces concepts permettrait ensuite de déterminer si la sensibilité est suffisante pour qu'une non-prédiction implique une absence (valeur $\{f\}$) et ainsi activer le mode "falsehood". 

Un des avantages de travailler avec les ensembles de valeurs de vérité est la possibilité d'introduire des valeurs quantitatives. La méthode mise en place actuellement est uniquement qualitative. On ne distingue pas, par exemple,le cas de deux variants d'une voie métabolique \texttt{A} et \texttt{B} avec le concept \texttt{A} représenté par trois ensembles $\{t\}$ et un ensemble $\{\emptyset\}$, et le concept \texttt{B} avec un ensemble $\{t\}$ et trois ensembles $\{\emptyset\}$. En effet, ces deux concepts sont représentés par l'ensemble des observations $\{\{t\},\{\emptyset\}\}$. Le calcul du degré de vérité n'est donc pas impacté par la proportion des sous-ensembles qu'ils représentent. Par conséquent, le raisonnement actuel ne peut pas choisir entre le concept \texttt{A} ou \texttt{B} bien que le premier semble le plus plausible. Ainsi, l'introduction d'un aspect quantitatif dans le calcul des degrés de vérité, de fausseté et d'incertitude permettrait de choisir plus finement un variant métabolique par rapport à d'autres variants ayant le même ensemble de vérité.

Dans l'objectif de compléter les annotations manquantes dans les génomes, des vecteurs de présence/absence de réactions pour des organismes ayant leurs voies métaboliques complètes pourraient être utilisés comme donnée de référence pour un système de recommandation. Ce système proposerait des réactions additionnelles permettant de combler les "trous" pour les organismes ayant des voies incomplètes. Ces nouvelles prédictions pourraient à leur tour être intégrées dans le raisonnement de GROOLS pour vérifier leur cohérence.

En ce qui concerne l'interface utilisateur, le raisonnement réactif mis en place permet de réagir à l'ajout et à la suppression  d'observations . Cette fonctionnalité pourrait être pleinement exploitée à travers une application (web ou "desktop") pour permettre aux biologistes d'interagir directement avec le graphe de connaissances. Il pourrait ainsi tester des théories et observer directement les conséquences logiques.

\end{refsegment}
