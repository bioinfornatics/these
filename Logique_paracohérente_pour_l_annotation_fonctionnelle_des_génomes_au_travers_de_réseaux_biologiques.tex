% !TeX encoding = utf8
% !TeX spellcheck = fr_FR
%\documentclass[a4paper,12pt,french,twoside,openright,openbib, final]{thesis} %draft, final%
\documentclass[univ]{thesis}

\author{Jonathan Mercier}
\org{Genoscope}
\thesisdirector{Claudine Médigue}
\thesiscodirector{David Vallenet}
\nationalthesisnumber{2017SACLE007}
\title{Logique paracohérente pour l’annotation fonctionnelle des génomes au travers de réseaux biologiques}
\subtitle{de la logique à la biologie\ldots}
\date{15 Mars 2017}
\location{Évry}
\discipline{Bio-Informatique}
\doctoralschoolnumber{577}
\doctoralschool{Structure et dynamique des systèmes vivants}
\keywords{biologie computationnelle, logique , métabolisme, représentation des connaissances, savoir, inconsistance, logique partiel}


\includeonly{glossaires,remerciement,introduction,part_1,part_2,part_3,annexe}


\begin{document}
    \maketitle
    \frontmatter
    \thispagestyle{plain}
    \chapter*{Remerciement}
    \vspace{-4cm}
    \handwritingFont
    Ce mémoire est le résultat d’un travail de recherche de près de trois ans, au sein du GenoScope. J'ai rencontré de nombreuses personnes que je souhaiterais faire part de ma reconnaissance.
    
    Je veux adresser tous mes remerciements aux personnes avec lesquelles j’ai pu échanger sur mes recherches et qui m’ont aidé pour la rédaction de ce mémoire.
    
    En commençant par remercier tout d’abord Claudine et David pour m'avoir donné la chance de faire cette thèse.
    
    Merci Claudine, de m'avoir accueilli dans ton laboratoire, de ton aide précieuse et pour le temps consacré à me améliorer ce mémoire.
    
    Merci David, j'ai aimé travailler avec toi, tu es passionné tout comme nos échanges, et souvent on était d'accords, mais on employés des mots différents. Retranscrire un raisonnement qui nous était intuitif n'a pas toujours était simple, alors quand il fallait l'adapter au métabolisme avec par endroit des règles humaines "mais ici c'est comme ça \ldots" ce fut très difficiles pour trouver la bonne approche logique effectué inconsciemment par nos raisonnements. On peut le dire, la problématique biologique nous a permis ensemble de relever ce challenge que l'on pensait résolu depuis longtemps par les logiciens. Grâce à nos échanges et nos entêtements mutuelles on a permis modestement d'apporter notre contribution à la science.
    Encore une fois merci pour tout David.
    
    Je remercie Alexandra, pour ses moments passer à discuter lors de nos bout de chemin en RER. Lors d'une thèse on rencontre inévitablement des moments difficiles et dans ces moments il a été agréables d'avoir quelqu'un avec qui discuter.
    
    Je pense également à Stéphane, je te remercie pour ces ballades à vélo, nos discussions techniques et ainsi que toutes les autres \ldots autant de bon moments passé ensemble.
    
    Je remercie Mark et Karine c'est toujours un plaisir de discuter avec vous et j'espère pouvoir échanger encore d'autre moments avec vous.
    
    Je te remercie Zoé, au premiers abords on peut avoir peur, tu martyrise les plantes avec un ciseau et du les empoisonnes avec du café. Bon! on a toujours pas de plante qui fait le café mais on découvre rapidement que tu est une personne gentille que l'on s'attache facilement.
    
    Forcément j'oublie pas "petit" David (mais grand par l'amitié que je te porte), travailleur et passionné, pendant longtemps tu était un des derniers à partir du laboratoire. D'ailleurs c'était à de tel moments que l'on avait le plus souvent l'occasion de discuter. Merci pour tous ses moment partagées.
    
    Un grand merci à toutes l'équipe du LABGeM, Aurélie pour sa bienveillance, Adrien pour nos nombreux échanges techniques, Alexandre avec qui on a partagé au début des problématiques similaires (nemometa, drools, java, maven et j'en passe \ldots) et à tous les autres.
    
    Je tiens à remercier également Claude Scarpelli pour ça bienveillance et nos diverses discussions abordée dans le RER.
    
    Un grand merci à jean-marc Aury de m'avoir accueilli dans son équipe. Tu m'as permis d'acquérir de l'expérience sur les thématiques de l'assemblage de séquences.
    
    Je pense également à Arnaud, Frédérick et Stefan on a passé de très bon moment, le GenoFoot, les soirées foot mais pas seulement. Ne changer rien il est agréable d'être à vos côté.
    
    Je remercie de façon plus large à toutes la grande famille du GenoScope, je vous souhaite à tous que du bonheur.
    
    Enfin, je tenais à remercier mes parents, mon petit frère, ma petite sœur et tous mes proches qui m'ont accompagné, aidé, soutenu et encouragé tout au long de la réalisation de ce mémoire. Et, plus que tout, merci à ma compagne, Lucie, pour son amour et son attention, toujours là, dans les hauts comme dans les bas, ce qui m'a permis de mené cette thèse avec succès.

%\vspace*{-1cm}
%\raggedleft \includegraphics[height=2cm]{img/merci.jpg}
\begin{textblock}{3}(14,24)
    \includegraphics[height=2cm]{img/merci.jpg}
\end{textblock}

\normalFont
    
    \printglossary[title={Abréviations}]
    \cleardoublepage
    
    \renewcommand{\contentsname}{Sommaire}
    {
    	\hypersetup{hidelinks}
	    \tableofcontents*
    }
    
    \mainmatter
    \pagestyle{thesisStyle}
    
    \begin{refsegment}
\chapter*{De l'information au savoir}
\markboth{Introduction --- De l'information au savoir}{Introduction --- De l'information au savoir}
\addcontentsline{toc}{chapter}{\color{MidnightBlue}\large\textsc\bfseries{Introduction --- De l'information au savoir}}
Avec l'avènement des outils informatiques, la quantité d'information publiée ne cesse d'augmenter. Cette abondance de données disponibles rend plus difficile la gestion de l'information. Une des premières personnes qui a évoqué cette problématique d'explosion de l'information, est Frank Fremont-Smith directeur de l'institut Américain des Sciences Biologiques,  en \citeyear{fremont61}  \cite{fremont61}. Cette problématique est toujours d'actualité. Par exemple depuis 2012, chaque année, plus de 2,8 millions de documents scientifiques sont publiés  \cite{oecd2016} . Face à cette arrivée massive de savoir, la vérification et le croisement de toutes les publications ne sont plus possibles.

De plus, une partie non négligeable de ces documents scientifiques traite de thématiques impliquant un très grand nombre de données. Ainsi, à ce nombre conséquent de publications, s'ajoute une quantité de données importantes et très hétérogènes.

Au regard de cette problématique, comment mettre en place une démarche scientifique pour vérifier nos savoirs ?

Une méthode utilisée jusqu'alors implique la vérification des théories par la mise en place d'expériences répétées. Cette méthodologie basée sur l'expérimentation permet de vérifier une hypothèse par des observations. L'expérimentation doit être menée de manière à ce que les conséquences étudiées soient liées de façon certaine à leur cause.

Il est important de rappeler que cette méthodologie a joué et joue toujours un rôle important dans les découvertes scientifiques. Cette méthode passe dans un premier temps par la formulation d'une hypothèse. Les différents acteurs de l'expérience sont déterminés. Puis dans un second temps, on établit le plan de l'expérience, pour terminer enfin sur l'évaluation des résultats obtenus. Afin de conforter les résultats, l'expérience est répétée. Ce processus permet d'apporter les éléments de confiance et d'impartialité nécessaires pour vérifier une théorie. Toutefois la méthodologie expérimentale s'avère souvent plus longue et plus onéreuse que les méthodes basées sur des prédictions \textit{in silico}. 

\note{Dès l'antiquité, Aristote (384 - 322 av. J.-C.) décrit la nécessité d'expliquer les causes par l'utilisation de conséquences liées et avérées. Mais la première personne reconnue pour avoir utilisée cette méthodologie est Alhazen (965 - 1039). A travers son traité \citetitle{Alhazen1572}, il montre, par des méthodes expérimentales que la lumière voyage en ligne droite. La découverte de cette loi physique est exceptionnelle pour l'époque.}

Les méthodes basées sur le calcul informatique prennent de plus en plus d'importance. Elles sont facilement réutilisables, rapides et capables de traiter un très grand nombre d'informations. Ces méthodes sont développées en utilisant un jeu d'information déterminé au départ. Une fois que les résultats de la méthode sont validés, nous pouvons rechercher un algorithme capable de généraliser et d'apporter une solution globale au problème. Rien ne garantit que les informations fournies par la suite respectent le cadre théorique posé par l'algorithme. C'est pourquoi les résultats obtenus sont considérés comme des prédictions. Ces résultats n'ont pas le même degré de “certification” que ceux obtenus par une observation empirique.

En conséquence, tant les prédictions informatiques que les hypothèses émises par l'Homme devraient, dans l'idéal, être vérifiées par une approche expérimentale. Afin de minimiser le nombre d'expérimentations, est-il intéressant de valider ou non des théories par comparaison à ce que l'on s'attend à obtenir vis-à-vis de ce que l'on prédit ?

Avec l'essor de ces nouvelles technologies numériques, la barrière entre information et savoir devient de plus en plus floue.

\note{
    L'origine du mot science, vient du latin \textit{"scientia"} désignant le savoir.
    
    La connaissance est propre à une personne. À contrario le savoir est transmissible.
    
    De nombreuses langues, comme l'anglais, ne possèdent pas de mot pour différencier savoir et connaissance. En effet, dans les deux cas on utilise \textit{"knowledge"}. Afin de marquer la nuance, on retrouve l'expression \textit{"certified knowledge"} pour le savoir et \textit{"learning by doing"} pour la connaissance.
}


Est-il possible de réunir les observations issues de méthodologies \textit{in silico} et celles issues de l'expérimentation en laboratoire autour d'un modèle théorique ?

Cette question nous amène à réfléchir à la représentation puis la validation des théories. En effet, comment évaluer des théories partiellement observées ? Mais également comment gérer les observations contradictoires ?

Apporter une méthode à cette vaste problématique permettrait de valider nos savoirs, d'identifier les concepts non observés et de caractériser les concepts contradictoires. Représenter ce savoir c'est également faciliter les échanges entre les différents domaines scientifiques.

\citation{La séparation des savoirs, la spécialisation en domaine isolé nuit considérablement au développement de la recherche.}{Jacques Le Goff}[Le Monde de l'éducation - mai 2000]

Ce constat se vérifie en biologie notamment avec l'avènement des séquenceurs de nouvelle génération. Le séquençage des organismes est devenu peu couteux et rapide. Ainsi la communauté scientifique a initié de vastes projets de séquençage de génomes qui visent à déterminer la séquence d' \gls{ADN} des organismes . Elle peut être composée de quelques milliers à plusieurs centaines de millions de paires de bases nucléiques. Selon l'ordonnancement des bases, des régions appelées gènes, confèrent des fonctionnalités à l'organisme. Ainsi, l'\gls{ADN} est un point de départ pour l'étude et la compréhension des fonctionnalités inscrites dans le vivant.


\begin{shadedfigure}
    \centering
    \includegraphics{img/simple_annotation_process.pdf}
    \caption{Vue globale du gène à l'annotation.}
    \label{fig:glob_annotation}
\end{shadedfigure}

Des outils bio-informatiques analysent ces séquences afin de prédire les régions géniques et leurs fonctions dans l'organisme (voir \cref{fig:glob_annotation}). Selon les organismes, le nombre de gènes varie de quelques centaines à plusieurs dizaines de milliers de gènes. Ainsi, parmi le million voire milliard de paires de bases d'\gls{ADN}, il faut identifier tous les gènes. Par conséquent, l'expertise humaine d'un génome est un défi en soi. Ces recherches se sont intensifiées et complexifiées, notamment avec la mise en place du projet d'étude de 100 000 génomes d'organismes pathogènes \cite{100kfoodborne}, ou encore l'analyse des éco-systèmes poly-microbiens présents sur l'Homme \cite{hmp}.

Pour se faire, des outils bio-informatiques ont automatisé le traitement de l'information issue des séquenceurs afin de traiter un nombre d'organismes toujours plus grand. Ces outils, alimentés continuellement en nouveaux génomes, ont amplifié le déluge d'information. Dans le domaine de l'annotation fonctionnelle (i.e. la prédiction de la fonction des gènes et des protéines), moins d'un pour cent des données ont pu être vérifiées (au regard des statistiques publiées par UniProt et SwissProt en 2017 \parencites{uniprot_stat}{expasy_stat} ). Ce fossé entre information de confiance et prédiction s'accélère car il est plus rapide de produire de l'information que de la vérifier.

De plus, les prédicteurs automatiques de la fonction biologique des gènes ne sont pas fiables. En effet, 30\% des annotations fonctionnelles seraient incorrectes, voire 80\% dans certaines familles de protéines \parencites{devos2001intrinsic}{schnoes2009annotation}. Ces séquences incorrectement annotées sont ensuite propagées dans les bases de connaissances.

Une des méthodes d'assignation de fonction de gènes consiste à inférer une annotation provenant d'un gène connu à toutes les séquences similaires. En effet, il est supposé dans l’évolution des organismes que les produits des gènes sont sous pression de sélection afin de préserver leur fonction. L'ensemble des fonctions d'un organisme permet à ce dernier d'être adapté à son environnement. Une modification d'un acide aminé impliqué dans la fonction d’une protéine peut potentiellement entraîner la perte de l'activité biologique de la protéine et par conséquent mettre en péril les capacités de survit de l'organisme. Ces régions d'acides aminés devraient faiblement variées d'un point de vue physico-chimique. Le transfert, de proche en proche des fonctions biologiques tend à surestimer les propriétés biologiques d'une séquence car les séquences annotées de façon erronée se retrouvent parmi les autres et sont potentiellement utilisées, afin de propager leur fonction à un autre gène considéré similaire, détériorant un peu plus la qualité des bases de données.

L'objectif de l'annotation des fonctions géniques est de fournir un catalogue des capacités moléculaires et/ou biochimiques dont est pourvu un organisme. Ce catalogue permet de mieux comprendre le vivant. Cependant, le processus d'annotation produit et amplifie l'assignation de fonctions erronées à de nouveaux gènes et entraîne l'incapacité à utiliser ces prédictions sans prendre un risque. Le catalogue de fonctions géniques est en effet utilisé par la suite dans de nombreux domaines, comme la biochimie, la biologie des systèmes avec l’étude des voies métaboliques et des réseaux de régulation, l'étude des écosystèmes etc. Cette problématique impacte notre compréhension du vivant et notre capacité à l'étudier, remettant en cause tout le processus d'annotation des gènes utilisé jusqu'alors.

Face à cette problématique, des approches variées ont été développées. On distingue les systèmes d'annotation automatique à base de règles, reprenant le raisonnement appliqué par les bio-curateurs, comme le projet \gls{HAMAP} \cite{lima2009hamap}. Ces règles sont généralement basées sur la séquence génomique et la taxonomie de l'organisme. Elles peuvent être créées par un bio-curateur ou par des outils d'apprentissage \cite{uniprot2011ongoing}.

On retrouve également les systèmes de reconstruction des voies métaboliques \cite{karpe2011pathway}. Ces méthodes utilisent des génomes complétement séquencés et bien annotés pour décrire des enchaînements de réactions amenant à des composés d'intérêt biologique.  Cette succession de réactions met en lumière des chemins à travers le réseau de réactions\footnote{D'où le terme anglais "pathway" pour décrire un chemin qui amène à un objectif biologique} qui sont appelés voies métaboliques. Ces réseaux de réactions forment un graphe de connaissance, spécifique de l'organisme. Toutefois, certaines réactions vont apparaître manquantes au regard des activités enzymatiques prédites à partir du génome de l'organisme. Le bio-curateur doit ainsi rechercher dans le génome des gènes candidats pour compléter les étapes d'une voie métabolique. Il est également en mesure d'estimer un degré de complétion des annotations fonctionnelles au regards de ses voies métaboliques prédites dans l'organisme.

Ce travail de curation des annotations par un bio-curateur est nécessaire mais reste une tâche fastidieuse, laborieuse et source d’erreurs. Il apparaît nécessaire de fournir un assistant à la curation de fonctions biologiques des gènes. Une méthode prenant en compte les prédictions \textit{in silico} de fonctions, mais également, le contexte biologique de ces fonctions, pourrait, à partir de règles se rapprochant du raisonnement humain, aider à l’expertise. De plus, elle permettrait de suggérer des annotations fonctionnelles ne pouvant pas être détectées par les prédicteurs usuels. 

\section*{La démarche suivie dans cette thèse}

Mon travail de recherche s'est ouvert à de nombreuses disciplines afin d'apporter une méthode d'expertise des observations vis-à-vis du savoir en biologie. En effet, la complexité du problème nécessite la mise en œuvre de concepts provenant de la logique, la représentation des connaissances, la théorie des graphes, la bio-informatique et le métabolisme. Ainsi, j'ai étudié ces différents domaines, décrypté le jargon, recherché les méthodes semblant avoir une application pour l'expertise de l'annotation fonctionnelle des génomes . Puis, j'ai combiné ces différents concepts dans le but d'obtenir une méthode fiable et rapide.

Certaines voies furent des impasses, d'autres n'avaient pas encore de solution. À travers ces quelques chapitres, je vous livre les notions et mon expérience sur ces différents domaines.

Cette thèse débute avec une problématique biologique \textit{"Comment guider et faciliter le travail des bio-curateurs lors des étapes de l'annotation fonctionnelle ?"}. Pour y répondre, nous avons souhaité reprendre un prototype de vérification de la cohérence globale de l'annotation développée dans l'équipe \texttt{HELIX} de l'Institut National de Recherche en Informatique et en Automatique dirigée par \textit{Alain Viari}. Leur projet a abouti à un raisonneur nommé \gls{HERBS}. Cet outil permet de discriminer les concepts biologiques attendus et correctement prédits des autres concepts. Mon travail de recherche a débuté par le souhait d'étendre les fonctionnalités de \gls{HERBS} afin de prendre en compte l'incertitude et la contradiction. C'est-à-dire d'avoir un raisonneur capable d'utiliser des observations évoquant l'affirmation et l'infirmation mais également le fait de ne pas savoir ou d'avoir des points de vue divergents. Au départ, nous pensions que le travail consisterait à récupérer les méthodes logiques, à les adapter à la biologie, puis mettre à jour l'outil. Or à aucun moment, nous nous doutions que certaines problématiques de la logique étaient toujours ouvertes. En effet, la logique a ses limites, notamment lorsque l'on travaille avec des concepts dont certains peuvent prendre des états ni-vrai-ni-faux ou encore, lorsque l'on souhaite raisonner sur des ensembles d'ensembles. Comme je l'ai compris plus tard, le monde de la logique a été rythmé par des faits marquants comme le paradoxe de \textit{Russell}. Tel un historien, je me suis retrouvé dans les grandes problématiques de la logique moderne. Ainsi j'ai suivi les traces de \textit{Platon} avec les bases de la logique classique, de \textit{Bertrand Russell} avec les ensembles, de \textit{Jan Łukasiewicz} avec le principe de tiers exclu et de \textit{Nuel Belnap} avec la logique à quatre valeurs.

\note{Le paradoxe de Russell démontre que si un ensemble est membre de lui-même, alors par définition il ne peut être un membre de lui-même. Mais s'il n'est pas un membre de lui-même, alors il est un membre de lui-même.
    
    Le paradoxe du barbier image une telle situation. Imaginez, le conseil municipal d'un village vote un arrêté municipal qui enjoint à son barbier (masculin) de raser tous les habitants masculins du village qui ne se rasent pas eux-mêmes et seulement ceux-ci.
    
    Le barbier, qui est bien un habitant du village, n'a pas pu respecter cette règle car :
    \begin{itemize}
        \item S'il se rase lui-même, il enfreint la règle, car le barbier ne peut raser que les hommes qui ne se rasent pas eux-mêmes ;
        \item S'il ne se rase pas lui-même - qu'il se fasse raser ou qu'il conserve la barbe - il est en tort également, car il a la charge de raser les hommes qui ne se rasent pas eux-mêmes.
    \end{itemize}
}

D'autre part, il est apparu très tôt la nécessité de représenter le savoir dans un modèle générique. Un tel modèle permet d'intégrer le savoir provenant d'un large éventail d'entrepôts de données. Cette recherche commença par les travaux de \textit{John F. Sowa} sur la représentation des connaissances ( \citeyear{sowa92,sowa99}). Pour cela, un travail de structuration des concepts et de classification de leurs relations a été mené. Ce travail m'a conduit à étudier la logique de description \cite{baader2003description}. Concrètement, ce domaine de recherche permet de faire le lien entre l'intelligence artificielle et la représentation du savoir. Les études menées sur cette représentation du savoir est un sujet très dynamique, notamment avec l'essor du \textit{Web Semantic}. Cette thématique se consacre à l'étude de la nature des concepts, de leurs relations, mais également de leur existence définissant par la même l'ontologie.

Dans l'objectif de cartographier notre savoir en Biologie, un consortium international s'est créé : \textit{\gls{GO}} (voir G. O. Consortium
et al \citeyear{go2001,go2004}). Les concepts sont classifiés parmi trois catégories distinctes (i) les fonctions moléculaires, (ii) les processus biologiques et (iii) les composants cellulaires. Les concepts sont appelés des GO termes. Les relations entre les termes possèdent des étiquettes pour exprimer les notions de composition et de type \cref{fig:go_relation}. Ces descriptions de la connaissance sont réalisées avec une grammaire et un vocabulaire restreints, afin de réduire l'ambiguïté, et donc la complexité des textes. Un tel langage permet à un programme informatique de comprendre une phrase. Ces  notions sont comprises dans l'expression "langage contrôlé". Ainsi cet ensemble de mots peut être utilisé pour vérifier la cohérence du texte ou encore de permettre à un utilisateur d'effectuer des interrogations sur l'ensemble des connaissances décrites.

\begin{shadedfigure}[H]
    \centering
    \includegraphics[width=0.7\textwidth]{img/dag_go_relation.png}
    \caption{Exemple de représentation de concepts et de leurs relations selon \acrfull{GO}.\hspace{\textwidth}\tiny{Source: \url{ftp://ftp.geneontology.org/pub/go/www/GO.ontology.relations.shtml}.}}
    \label{fig:go_relation}
\end{shadedfigure}


La structure des données de \gls{GO} répond à nos besoins. Cependant, les liens entre les termes d'une catégorie avec une autre catégorie ne sont pas fournis. Par exemple, on ne peut pas relier des processus biologiques à des fonctions moléculaires. Des projets comme \cite{AdditionalGO2006} parviennent partiellement à couvrir les liens  entre les termes des différentes catégories. Toutefois, le nombre de relations manquantes reste élevé. Pour cette raison, j'ai mis au point une structure de concepts contenant toutes les notions nécessaires pour représenter les connaissances à utiliser. Cette structure doit devenir le support d'inférence des observations biologiques, l'objectif étant de vérifier l'existentialité de nos connaissances sur chaque organisme.

Tout au long de mon travail de recherche, j'ai été confronté à des limites, si bien que des solutions nouvelles devaient être proposées. Des problématiques d'apparence simple se sont révélées plus complexes. J'ai ainsi dû rechercher des solutions dans des domaines "éloignés" de la bio-informatique comme \textit{la Logique}, afin de définir un modèle générique représentant toute sorte de données, même celles auxquelles je n'avais pas pensées ! J'ai également mis en place des méthodes permettant d'intégrer un volume de données conséquent tout en étant capable de fournir un résultat dans un temps raisonnable. Ce travail m'a demandé de dépasser les paradoxes de la logique afin de l'étendre aux notions d'"inconnu" et de "contradiction" ; la méthode finale doit proposer des annotations manquantes, mettre en lumière des contradictions, vérifier les prédictions \textit{in silico} avec les "expectations" biologiques (voir Note ci-dessous), et explorer les capacités métaboliques de tout organisme.

\note{Ce travail de recherche s'attache à prendre en compte les fonctions biologiques attendues dans un organisme. Ce savoir peut provenir de fait établis, de suppositions ou autres. Lorsque nous sommes confronté à de tels faits, nous espérons trouver une théorie qui corrèle les faits observés. Pour exprimer cette idée, je vais utiliser au long de cette thèse le terme expectation. C'est un mot transparent. Il peut être utilisé en français et en anglais. Il vient du latin \textit{exspectatio} \cite[p. 637]{gaffiot1934dictionnaire} et permet d'exprimer une anticipation.\\\tiny{sources:\\ \url{http://www.larousse.fr/dictionnaires/francais/expectation/32223}\\\url{https://fr.wiktionary.org/wiki/exspectatio} } }

Dans la suite de cette thèse, je vais tout d’abord expliciter les concepts métaboliques et leurs représentations informatiques. Puis, les liens entre génome et métabolisme seront détaillés, avant de s'intéresser aux données biologiques qui sont à notre disposition pour l’analyse de fonctions telles que les réseaux métaboliques. La section suivante abordera des notions de \textit{Logique} ainsi que des méthodes existantes aujourd'hui pour résoudre des problèmes biologiques. Je présenterai ensuite mes investigations sur la logique paracohérente et les premières phases du développement d'un système expert nommé \gls{GROOLS}. Le dernier chapitre porte sur le raisonnement descriptif finalement mis en œuvre dans la méthode \gls{GROOLS} et la présentation de résultats. Enfin, cette thèse se termine par des perspectives de recherche.



\end{refsegment}

    
    \chapter*{De l'information à la connaissance}
\markboth{Introduction --- De l'information à la connaissance}{De l'information à la connaissance}
\addcontentsline{toc}{chapter}{Introduction --- De l'information à la connaissance}

Un des enjeux majeur de la science, a été de tout temps, la recherche de la véracité des connaissances. En effet, une connaissance communément acceptée, permet d'aboutir à un savoir transmissible. Pour cela, il est indispensable de procéder avec méthode.

Dès l'antiquité, la nécessité de démontrer, les causes expliquant avec certitudes les conséquences, émerge avec Aristote (384 - 322 av. J.-C.). À travers Alhazen (965 - 1039) l'idée pris forme autour de méthodes expérimentales et reproductibles. Son traité \citetitle{Alhazen1572} détaille des expériences reproductibles, ainsi tout lecteur, parvient aux mêmes conclusions. Ses méthodes furent reprises en Occident par Roger Baconn (1220 - 1294). Cette démarche expérimentale décrit les prémisses d'une méthode scientifique. Elle a évolué tout au long de l'histoire, afin d'être rigoureuse, vérifiable et reproductible.

Ainsi, le trio observation, modélisation et expérimentation s'est imposé comme fondement d'une démarche scientifique. Ces trois paramètres permettent d'amener les éléments de confiance et d'impartialité nécessaire pour vérifier une connaissance.

Cependant l'explosion des connaissances a rendu cette démarche fastidieuse. En effet, elle nécessite une intervention humaine minutieuse de toutes les observations vis-à-vis du savoir. Par exemple le nombre de communications scientifique, est estimé à plus 50 millions \citep[voir][]{LEAP:LEAP0509}. C'est autant de connaissances à confronter  On peut légitimement se demander, s'il est possible de confronter l'ensemble des observations, vis-à-vis d'un ensemble de connaissance. Ceci afin d'évaluer nos certitudes et produire in-fine du savoir.

\citation{
	Nous estimons posséder la science d'une chose d'une manière absolue, \ldots quand nous croyons que nous connaissons la cause par laquelle la chose est, que nous savons que cette cause est celle de la chose, et qu'en outre il n'est pas possible que la chose soit autre qu'elle n'est.}{Aristote}[(Seconds Analytiques I, 31, 88a, 4)]

Les observations sont des éléments précurseurs à la connaissance, elles sont utilisés pour émettre des théories. L'émergence des technologies de l'information nous permettent de récupérer un nombre d'observations toujours plus grand. Cet évènement ouvre de nouvelles possibilité comme l'étude massive des observations scientifiques. C'est l'ère du "Big Data". Notre capacité à générer des informations est sans cesse grandissante. Elle dépasse largement les capacités humaines nécessaires à l'expertise de ces informations. En effet, les outils de mesure génèrent continuellement et rapidement des observations. Ces diverses observations brutes sont ensuite utilisées par des méthodes informatisées afin d'émettre des prédictions issues de ces observations primaires. A développer? ou veut on en arriver?

Les super-calculateurs, support de ces méthodes informatisées, deviennent de plus en plus rapide, augmentant les capacités de traitement de l'information. De tels outils, se démocratisent dans tous les domaines de la science. Pour autant seul une partie de ces informations peuvent être étudiée afin de créer de nouvelles connaissances. Avec l'essor des nouvelles technologies, la barrière entre information et connaissance devient de plus en plus flou.

\note{Il est intéressant de remarquer que le mot science vient du latin \textit{"scientia"} désignant la connaissance.}

Ces données produites par la recherche sont entreposées dans des bases de données variées, adaptées aux différentes spécialités. Certains entrepôts de données peuvent couvrir un même champs de connaissance. Ne disposant pas des mêmes données, tantôt en la complétant, tantôt en la contredisant, laissant même des régions de connaissances couverte par aucune donnée, appelé "trou de connaissance ".

Il est indubitable que nous ne pouvons plus mener des recherches sur des vastes domaines comme "la Biologie", "les Mathématiques", "la Physique". Les connaissances à maitriser sont trop nombreuses. Les chercheurs, se spécialisent de plus en plus afin de maitriser un domaine toujours plus pointu. Malgré tous ces efforts, dans de nombreux domaines, il nous est impossible d'appréhender tous les travaux liés. Or ces connaissances, nous sont non seulement
importantes pour valider ou non les prédictions ; mais il nous est également nécessaire d'élargir nos connaissances, hors de notre domaine d'expertise pour mieux le comprendre en retour.(trop long)

Rechercher la véracité des connaissances, nécessite de pouvoir représenter nos supposées connaissances puis de les confrontés aux différentes prédictions et expectations. Ce travail ne peut plus être réalisé uniquement par l'Homme, il doit être assisté par des méthodes informatisées capable de répondre aux défis d'aujourd'hui et de demain.

Ce constat se vérifie en biologie, avec l'avènement des séquenceurs nouvelles générations. Le séquençage des organismes est devenu abordable et rapide. Ainsi la communauté scientifique à initier de vaste projet de séquençage du monde vivant. Ces projets permettent d'avoir le code génétique, également nommée séquence \gls{ADN}.L'ADN peut faire quelques centaines de milliers à plusieurs centaines de millions de paires de bases nucléiques. Cette séquence nucléotidique est un point départ pour l'étude et la compréhension des fonctionnalités inscrite dans le vivant.

Des outils bio-informatiques analysent ces séquences afin de prédire les régions géniques et leurs fonctions dans l'organisme. Le nombre de gènes peut aller de quelques centaines à plusieurs dizaines de milliers selon les organismes. Par conséquent, l'expertise humaine d'un génome est un défi en soi. Ces recherches se sont intensifiées et complexifiées, comme l'étude de 100 000 génomes d'organismes pathogènes \citep[voir][]{100kfoodborne}, ou encore sur les eco-systèmes poly-microbiens présent sur l'Homme \citep[voir][]{hmp}.

Pour se faire des outils bio-informatiques ont automatisé le traitement de l'information issue des séquenceurs afin de traiter un nombre d'organisme toujours plus grand. Ces outils, alimentés continuellement en nouveaux génomes, ont amplifié le déluge d'information. Dans le domaine de l'annotation fonctionnelle, moins d'un pour cent des données ont pu être vérifiées (au regard des statistiques publiés par UniProt et SwissProt \parencites{uniprot_stat}{expasy_stat} ). Ce fossé entre information de confiance et prédiction s'accélère car il est plus rapide de produire de l'information que de la vérifier.

Or les prédicteurs automatiques de fonctions géniques ne sont pas fiables. En effet, 30\% des annotations fonctionnelles seraient incorrectes, voire 80\% dans certaines familles de protéines \parencites{devos2001intrinsic}{schnoes2009annotation}. Ces séquences incorrectement annotées sont ensuite propagées dans les bases de connaissances.

Une des méthodes d'assignations de fonction a un gène, consiste à rechercher une séquence proche, dont la fonction est connue. Supposant qu'ils partagent une même fonction.(a reformulé) Ainsi on inférera l'annotation, à la séquence de fonction inconnue, car elle est proche de la première. Cette pratique tend à amplifier l'inexactitude des fonctions géniques. En effet, les séquences mal annotées se retrouvent parmi les autres et sont donc potentiellement ré-utilisées \unsure{est-ce utile de remettre une couche?} afin de propager une fonction d'un autre gène considéré proche. Détériorant un peu plus la qualité de ces bases de données.

L'objectif de l'annotation des fonctions géniques est de fournir un catalogue des capacités moléculaires et/ou biochimiques dont est pourvu un organisme. Ce catalogue permet de mieux comprendre le vivant.( a reformulé) Mais lorsque le processus d'annotation, produit et amplifie l'assignation de mauvaise fonction aux gènes. Cela entraine l'incapacité à utiliser ces prédictions sans prendre un risque. De plus, ce catalogue de fonctions géniques est utilisé par la suite dans de nombreux domaines, comme l'étude des voies métaboliques, la biologie des systèmes, la classification des gènes essentiels et autres. Cette problématique impacte notre compréhension du vivant et notre capacité à l'étudier, remettant en cause tout le processus d'annotation des gènes utilisés jusqu'alors.

Face à cette problématique des approches variées ont été développées. On distingue les systèmes d'annotations automatiques à base de règle, reprenant le raisonnement appliqué par les bio-curateurs, comme le projet HAMAP \citep[voir][]{lima2009hamap}. Ces règles sont généralement basé sur la séquence génomique et la taxonomie de l'organisme. Elles peuvent être créées par un bio-curateur ou par des outils d'apprentissage \citep[voir][]{uniprot2011ongoing}.

On retrouve également les systèmes de reconstruction des voies métaboliques \citep[voir][]{karpe2011pathway}, utilisant les prédictions de fonction génique, spécifiques d'un organisme afin de proposé des voies métaboliques décrites dans d'autres organismes. Ces prédictions forment un graphe de connaissance, spécifique de l'organisme. Ainsi, des fonctions marquées comme manquante à la réalisation des voies métaboliques sont suggérées aux bio-curateurs.

Les méthodes à bases de règles automatise l'annotation fonctionnelle par l'utilisation de règle lié à la biologie de l'organisme et non plus uniquement par une prédiction in-silico. Les méthodes utilisant la représentations des connaissances biologiques permettent de contextualiser les prédictions et de suggérer des annotations fonctionnelles ne pouvant être détecté in-silico. Toutefois le travail de curations des annotations par un bio-curateur est nécessaire, mais considéré comme fastidieux, laborieux et source d'erreur. Il est nécessaire de fournir un assistant à la curation des fonctions géniques.

%L'équipe HELIX dirigé par Alain Viari (INRIA) à développé un prototype de vérification de la cohérence globale de l'annotation (HERBS). Reprenant les systèmes à base de règle et la représentation des connaissances. Ainsi les voies métaboliques peuvent être rattaché à des caractères phénotypique. Par exemple, la pousse d'un organisme sur un milieu avec une seule source de carbone, un antibiotique \ldots peuvent être relié au graphe de connaissance et ainsi comparé les prédictions par rapports aux attentes issues des données expérimentales. 

%Cet discipline consiste à caractériser la fonction des gènes, afin de lister les fonctionnalités dont est pourvu un organisme. En amont de ce travail les outils bio-informatiques émettent des prédictions sur les fonctions des gènes identifiés. Toutefois seulement une partie de ces prédictions sont vérifiées expérimentalement ou par un curateur. Or de nombreuses études montre les limites de l'annotation fonctionnelle automatiques. 

\section*{La démarche suivie dans cette thèse}
\begin{refsection}
\chapter{Contexte biologique et méthodologique }

\section{Le métabolisme et sa représentation informatique}
\subsection{Généralités sur le métabolisme}
\subsection{Les acteurs}
\subsection{Représentation en graphe}
\subsection{Ressources sur les voies métaboliques}
\subsubsection{KEGG}
\subsubsection{Reactome}
\subsubsection{Unipathway}
\subsubsection{Genome properties}

\section{Des génomes aux réseaux métaboliques}
\subsection{Annotation fonctionnelle}
\subsection{Reconstruction des réseaux métaboliques}
\subsection{Modèles métaboliques}
\subsection{Les données expérimentales}
\subsubsection{Élucidation des voies métaboliques}
\subsubsection{Phénotypes de croissances}

\section{Raisonnement logique dans le processus de curation}
\subsection{Lacunes et incertitudes dans nos connaissances}
\subsubsection{Les trous dans les connaissances et les enzymes orphelines}
\subsubsection{Limites de l’annotation fonctionnelle et rôle de la curation}
\subsection{Logique et raisonnement}
\subsubsection{Les différentes logiques}
\paragraph{Logique booléenne}
\paragraph{Logique multi-valuée}
\subsubsection{Inférence d’information}
\paragraph{Représenation des connaissances/ontologies}
\paragraph{Chainage avant et arrière} %backward/forward
\paragraph{Règles et système expert}

\section{Méthodes existantes}
\subsection{HAMAP et UniRule}
\subsection{Genome properties}
\subsection{IMG terms}
\subsection{HERBS} \citep[voir][]{lima2009hamap}
\printbibliography[segment=\therefsegment,heading=subbibliography]
\end{refsection}
    
    Question principale:\\
Compte tenu de l'augmentation constante et rapide des annotations automatiques, un système expert peut il détecter les annotations inconsistantes et informer les possibles annotations manquantes vis à vis d'un organisme prokaryotes? 

idée directrices:
l'inférence logiques des prédictions peut se faire via une représentation en graphes des connaissances et l'utilisation de la logique non classiques permettant de représenter l'inconnu et la contradiction.

histoires:

herbs
   représentation en dag des connaissances
   description manuel des connaissances
   and or graph 
   logique classique T / F
   prédiction et expectation
   conclusion
   
Mais descriptions des connaissances fastidieuses .…… 
les prédictions inconnu sont considéré absentes suivant le principe du tiers exclut
ne peut pas gérer la contradictions..
structurations de l'information sous forme de phrases

vers une représentation des connaissances orienté-objets
vers une logique à 4 valeurs T,F,B,N pour gérer l'inconnu et la contradiction

belnap et ses tables de vérité

oui mais certains résultat semble incohérent  N ou B donc T ...

vers une représentation ensemblistes des connaissances et de leur valeurs de vérité
  - les ensembles et leur représentation en graphe conceptuelle
  - les valeurs de vérité généralisés

\chapter{Raisonnement logique dans un monde pas si logique}
\section{logique booleenne}
\section{logique à quatre valeurs}
\section{Le début de GROOLS}
expliqué les raté
\section{vers un raisonnement descriptif}
ensembliste valeur de vérité généralisé
\section{la méthode}
le papier

    \begin{refsegment}
\chapter*{Perspectives}
\addcontentsline{toc}{chapter}{Perspectives pour GROOLS}

Ces trois années de recherche m'ont permis de mettre au point un système expert, nommé \texttt{\gls{GROOLS}}, pour l'annotation fonctionnelle de génomes au travers de processus biologiques comme les voies métaboliques. Ces travaux ont tout d'abord consisté à choisir une représentation des connaissances et un raisonnement adaptés à la structure hiérarchique des données biologiques. En effet, une utilisation simple de la logique paracohérente à trois ou quatre valeurs de vérité ne permet pas de répondre complétement à la problématique. J'ai donc utilisé une représentation de l'information sous la forme d'ensembles de valeurs de vérité permettant d'obtenir une meilleure description des observations attachées aux différents concepts. La structuration hiérarchique des connaissances \textit{a priori} (théories) mise en place permet d'intégrer toutes ressources dont les concepts sont reliés par des liens de composition et/ou de généralisation. Cette représentation permet notamment d'utiliser les connaissances sur les réseaux métaboliques  de \textit{Genome Properties} et d'\textit{UniPathway} qui peuvent être étendues sans grande difficulté à d'autres ressources comme MetaCyc. L'utilisation des ensembles de valeurs de vérité permet au raisonneur de suggérer des variants de voies métaboliques avec des données biologiques qui peuvent être incomplètes et contradictoires. Le biologiste peut ainsi rapidement évaluer le niveau de complétion de l'annotation génomique. De plus, le système permet de repérer rapidement les réactions manquantes et les observations ambiguës. L'outil \texttt{GROOLS} est un "trait d'union" entre la bio-informatique et la biologie. Il est capable d'utiliser intelligemment les connaissances du domaine, les prédictions bio-informatiques et les résultats expérimentaux effectués en laboratoire afin de valider les théories.

L'outil a été évalué au travers de l'analyse de 14 génomes dont les organismes disposaient de données de phénotypes de croissance. La qualité originale du raisonnement de \texttt{\gls{GROOLS}} en fait un outil d'intérêt pour l'annotation des génomes mais l'outil \texttt{\gls{GROOLS}} pourrait également être utilisé dans d'autres domaines disposant de connaissances hiérarchiques.

Le système de représentation des connaissances peut être étendu selon deux axes. Le premier axe consiste à intégrer l'information de phénotype dans le graphe de connaissances. Dans l'étude précédemment menée sur 14 génomes bactériens, une observation de phénotype de croissance était directement associée à la voie dégradation correspondante. L'introduction de concepts de type phénotype dans le graphe de connaissances permettrait de mettre en relation un phénotype avec de multiples voies métaboliques. De plus, l'ajout d'un nouveau type de relation (i.e. "avoid" pour interdit) autoriserait la formulation de phénotypes plus complexes. Par exemple, pour une bactérie aérobie stricte, on ne s'attend pas à retrouver de voies anaérobies. Ainsi une expectation ($\{t\}$) sur le concept aérobie strict a pour conséquence logique d'inférer l'ensemble $\{f\}$ sur le concept anaérobie par une relation "avoid", en plus de la valeur $\{t\}$ sur le concept aérobie. Un second objectif serait d'associer des concepts "prédicteurs" à chaque unité fonctionnelle (e.g. réactions) à l'image de Genome Properties. Un attribut additionnel sur ces concepts permettrait ensuite de déterminer si la sensibilité est suffisante pour qu'une non-prédiction implique une absence (valeur $\{f\}$) et ainsi activer le mode "falsehood". 

Un des avantages de travailler avec les ensembles de valeurs de vérité est la possibilité d'introduire des valeurs quantitatives. La méthode mise en place actuellement est uniquement qualitative. On ne distingue pas, par exemple,le cas de deux variants d'une voie métabolique \texttt{A} et \texttt{B} avec le concept \texttt{A} représenté par trois ensembles $\{t\}$ et un ensemble $\{\emptyset\}$, et le concept \texttt{B} avec un ensemble $\{t\}$ et trois ensembles $\{\emptyset\}$. En effet, ces deux concepts sont représentés par l'ensemble des observations $\{\{t\},\{\emptyset\}\}$. Le calcul du degré de vérité n'est donc pas impacté par la proportion des sous-ensembles qu'ils représentent. Par conséquent, le raisonnement actuel ne peut pas choisir entre le concept \texttt{A} ou \texttt{B} bien que le premier semble le plus plausible. Ainsi, l'introduction d'un aspect quantitatif dans le calcul des degrés de vérité, de fausseté et d'incertitude permettrait de choisir plus finement un variant métabolique par rapport à d'autres variants ayant le même ensemble de vérité.

Dans l'objectif de compléter les annotations manquantes dans les génomes, des vecteurs de présence/absence de réactions pour des organismes ayant leurs voies métaboliques complètes pourraient être utilisés comme donnée de référence pour un système de recommandation. Ce système proposerait des réactions additionnelles permettant de combler les "trous" pour les organismes ayant des voies incomplètes. Ces nouvelles prédictions pourraient à leur tour être intégrées dans le raisonnement de GROOLS pour vérifier leur cohérence.

En ce qui concerne l'interface utilisateur, le raisonnement réactif mis en place permet de réagir à l'ajout et à la suppression  d'observations . Cette fonctionnalité pourrait être pleinement exploitée à travers une application (web ou "desktop") pour permettre aux biologistes d'interagir directement avec le graphe de connaissances. Il pourrait ainsi tester des théories et observer directement les conséquences logiques.

\end{refsegment}

    
    \cleardoublepage
    
    \backmatter
    {
        \cleardoublepage
%        \appendix
%        \thispagestyle{empty}
%        \chapter{Annexe}

\begin{lstlisting}[basicstyle=\tiny\normalfont\ttfamily,caption=data/processes/lysine\_DAP\_biosynthesis.data]
;;;; -------------------------------------------------------
;;; HERBS (Hamap Expert Rules Based System)
;;;
;;; @file: lysine\_DAP\_biosynthesis.data
;;; -------------------------------------------------------
;;;
(process declare lysine\_DAP\_biosynthesis present in ALL)
(process define lysine\_DAP\_biosynthesis -> and UPA00034)
(process define UPA00034 -> or UPA00034-alt-0 UPA00034-alt-1 UPA00034-alt-2 )
(process define UPA00034-alt-0  -> and ULS00006 ULS00007 ULS00009 ULS00010 ULS00011)
(process define UPA00034-alt-1  -> and ULS00006 ULS00008 ULS00009 ULS00010 ULS00011)
(process define UPA00034-alt-2  -> and ULS00006 ULS00227 ULS00009 ULS00010 ULS00011)
(process define ULS00006 -> or ULS00006-alt-0 ULS00006-alt-1 )
(process define ULS00006-alt-0  -> and UER00015 UER00016 UER00017)
(process define ULS00006-alt-1  -> and UER00015 UER00018 UER00017)
(process define ULS00007 -> and ULS00007-alt-0 )
(process define ULS00007-alt-0  -> and UER00019 UER00020 UER00021)
(process define ULS00009 -> and ULS00009-alt-0 )
(process define ULS00009-alt-0  -> and UER00025)
(process define ULS00010 -> and ULS00010-alt-0 )
(process define ULS00010-alt-0  -> and UER00026)
(process define ULS00011 -> and ULS00011-alt-0 )
(process define ULS00011-alt-0  -> and UER00027)
(process define ULS00006 -> or ULS00006-alt-0 ULS00006-alt-1 )
(process define ULS00006-alt-0  -> and UER00015 UER00016 UER00017)
(process define ULS00006-alt-1  -> and UER00015 UER00018 UER00017)
(process define ULS00008 -> and ULS00008-alt-0 )
(process define ULS00008-alt-0  -> and UER00022 UER00023 UER00024)
(process define ULS00009 -> and ULS00009-alt-0 )
(process define ULS00009-alt-0  -> and UER00025)
(process define ULS00010 -> and ULS00010-alt-0 )
(process define ULS00010-alt-0  -> and UER00026)
(process define ULS00011 -> and ULS00011-alt-0 )
(process define ULS00011-alt-0  -> and UER00027)
(process define ULS00006 -> or ULS00006-alt-0 ULS00006-alt-1 )
(process define ULS00006-alt-0  -> and UER00015 UER00016 UER00017)
(process define ULS00006-alt-1  -> and UER00015 UER00018 UER00017)
(process define ULS00227 -> and ULS00227-alt-0 )
(process define ULS00227-alt-0  -> and UER00466)
(process define ULS00009 -> and ULS00009-alt-0 )
(process define ULS00009-alt-0  -> and UER00025)
(process define ULS00010 -> and ULS00010-alt-0 )
(process define ULS00010-alt-0  -> and UER00026)
(process define ULS00011 -> and ULS00011-alt-0 )
(process define ULS00011-alt-0  -> and UER00027)
\end{lstlisting}

        \nocite{*}
        \thispagestyle{empty}
        \printbibliography
        \cleardoublepage
    }
    
    \newpage
    
    \thispagestyle{empty}
    \newgeometry{left=1.5cm,right=1.5cm,top=4cm,bottom=0.5cm}
    \begin{textblock}{3}(1,1.7)
        \includegraphics[height=2cm]{img/logo_ecole_doctorale.png}
    \end{textblock}
    \begin{mdframed}[linecolor=psviolet, linewidth=2pt, innerleftmargin=10, innerrightmargin=30, innertopmargin=10, innerbottommargin=50, font=\tiny]
        {\small \textbf{Logique paracohérente pour l’annotation fonctionnelle des génomes au travers de réseaux biologiques}}
        
        \noindent\textbf{Mots clés :} biologie computationnelle, logique , métabolisme, représentation des connaissances, inconsistance, logique partiel, raisonnement déductif
        
        \begin{multicols}{2}            
            \paragraph*{\textbf{Résumé:} }Face à l’augmentation des capacités de séquençage, on assiste à une accumulation de prédictions \textit{in silico} dans les banques de séquences biologiques. Cette masse de données dépasse nos capacités d’expertise humaine et, malgré des progrès méthodologiques, ces analyses automatisées produisent de nombreuses erreurs notamment dans la prédiction de la fonction biologique des protéines. Par conséquent, il est nécessaire de se doter d’outils capables de guider l’expertise humaine par une évaluation des prédictions en confrontation avec les connaissances sur l’organisme étudié. 
            
            GROOLS (pour "Genomic Rule Object-Oriented Logic System") est un système expert capable de raisonner à partir d’informations incomplètes et contradictoires. Il a été développé afin de devenir l’assistant du biologiste dans un processus d’annotation fonctionnelle de génome intégrant une grande quantité d’information de sources diverses. GROOLS utilise une représentation générique des connaissances sous la forme d’un graphe de concepts qui est orienté et acyclique. Les concepts représentent les différents composants de processus biologiques (e.g. voies métaboliques) et sont connectés par des relations de différents types (i.e. "part-of", "subtype-of").  Ces "Connaissances-a-priori" représentent des théories dont on souhaite élucider leur présence dans un organisme. Elles vont servir de socle au raisonnement afin d’être évaluées à partir d’observations de type "Prédiction" (e.g. activités enzymatiques prédites) ou "Expectation" (e.g. phénotypes de croissance). Pour cela, GROOLS met en œuvre une logique paraconsistante sur des ensembles de faits que sont les observations. Au travers de différentes règles, les valeurs de "Prédiction" et "d’Expectation" vont être propagées dans le graphe sous la forme d’ensembles de valeurs de vérité. A la fin du raisonnement, une conclusion sera donnée pour chaque "Connaissance-a-priori" en combinant leur valeurs de "Prédiction" et d' "Expectation". Les valeurs de conclusion peuvent, par exemple, indiquer une "Présence-confirmée" (i.e. fonction prédite et attendue), une "Absence" (i.e. fonction non prédite mais attendue) ou une "Présence-non-attendue" (i.e. fonction prédite mais pas attendue dans l’organisme).
            
            Le raisonnement de GROOLS a été appliqué sur plusieurs organismes microbiens avec différentes sources de "Prédictions" (i.e. annotations d’UniProtKB ou de MicroScope) et de processus biologiques (i.e. GenomeProperties et UniPathway). Pour les "Expectations", des données de phénotype de croissance et les voies de biosynthèse des acides aminés ont été utilisées. Les résultats obtenus permettent rapidement d’évaluer la qualité globale des annotations d’un génome et de proposer aux biologistes des annotations à compléter ou à corriger car contradictoires. Plus généralement, le logiciel GROOLS peut être utilisé pour l’amélioration de la reconstruction du réseau métabolique d’un organisme qui est une étape indispensable à l’obtention d’un modèle métabolique de qualité.
        \end{multicols}
    \end{mdframed}
    
    \begin{textblock}{9}(1.5,26.5)
        \begin{SingleSpace}
            {\tiny \noindent\textcolor{psviolet}{\textbf{Université Paris-Saclay} \\
                    Espace Technologique / Immeuble Discovery \\
                    Route de l’Orme aux Merisiers RD 128 / 91190 Saint-Aubin, France}}
        \end{SingleSpace}
    \end{textblock}
    \begin{textblock}{3}(17,25)
        \includegraphics[height=3cm]{img/univ-e.png}
    \end{textblock}
    
    \clearpage
    
    \thispagestyle{empty}
    \newgeometry{left=1.5cm,right=1.5cm,top=4cm,bottom=0.5cm}
    \begin{textblock}{3}(1,1.7)
        \includegraphics[height=2cm]{img/logo_ecole_doctorale.png}
    \end{textblock}
    \begin{mdframed}[linecolor=psviolet, linewidth=2pt, innerleftmargin=10, innerrightmargin=30, innertopmargin=10, innerbottommargin=50, font=\tiny]\textenglish{   
        {\small \textbf{Functional genomic annotation with paraconsistent logic through biological network}}
        
        \noindent\textbf{Keywords :} computationnal biology, logic , métabolism, knowledge representation, inconsistencies, partial logic, deductive reasoning
        
        \begin{multicols}{2}          
            \paragraph*{\textbf{Abstract:} }One consequence of increasing sequencing capacity is the the accumulation  of \textit{in silico} predictions in biological sequence databanks. This amount of data exceeds human curation capacity and, despite methodological progress, numerous errors on the prediction of protein functions are made.  Therefore, tools are required to guide human expertise in the evaluation of bioinformatics predictions taking into account background knowledge on the studied organism.
            
            GROOLS (for “Genomic Rule Object-Oriented Logic System”) is an expert system that is able to reason on incomplete and contradictory information. It was developed with the objective of assisting biologists in the process of genome functional annotation by integrating high quantity of information from various sources. GROOLS adopts a generic representation of knowledge using a directed acyclic graph of concepts that represent the different components of a biological process (e.g. a metabolic pathway) connected by two types of relations (i.e. “part-of” and “subtype-of”). These concepts are called “Prior Knowledge concepts” and correspond to theories for which their presence in an organism needs to be elucidated. They serve as basis for the reasoning and are evaluated from observations of “Prediction” (e.g. a predicted enzymatic activity) or “Expectation” (e.g. growth phenotypes) type. Indeed, GROOLS implements a paraconsistent logic on set of facts that are observations. Using different rules, “Prediction” and “Expectation” values are propagated on the graph as sets of truth values. At the end of the reasoning, a conclusion is given on each “Prior Knowledge concepts” by combining “Prediction” and “Expectation” values. Conclusions may, for example, indicate a “Confirmed-Presence” (i.e. the function is predicted and expected), a “Missing” concept (i.e. the function is expected but not predicted) or an “Unexpected-Presence” (i.e. the function is predicted but not expected in the organisms).
            
            GROOLS reasoning was applied on several organisms and with different sources of “Predictions” (i.e. annotations from UniProtKB or MicroScope) and biological processes (i.e. GenomeProperties and UniPathway). For “Expectations”, growth phenotype data and amino-acid biosynthesis pathways were used. GROOLS results are useful to quickly evaluate the overall annotation quality of a genome and to propose annotations to be completed or corrected by a biocurator. More generally, the GROOLS software can be used to improve the  reconstruction of the metabolic network of an organism which is an essential step in obtaining a high-quality metabolic model.
        \end{multicols}
    }
    \end{mdframed}

    \begin{textblock}{9}(1.5,26.5)
        \begin{SingleSpace}
            {\tiny \noindent\textcolor{psviolet}{\textbf{Université Paris-Saclay} \\
                                 Espace Technologique / Immeuble Discovery \\
                                 Route de l’Orme aux Merisiers RD 128 / 91190 Saint-Aubin, France}}
        \end{SingleSpace}
    \end{textblock}
    \begin{textblock}{3}(17,25)
        \includegraphics[height=3cm]{img/univ-e.png}
    \end{textblock}
    
    
\end{document}