\chapter*{Remerciement}
    \vspace{-4cm}
    \handwritingFont
    Ce mémoire est le résultat d’un travail de recherche de près de trois ans, au sein du GenoScope. J'ai rencontré de nombreuses personnes que je souhaiterais faire part de ma reconnaissance.
    
    Je veux adresser tous mes remerciements aux personnes avec lesquelles j’ai pu échanger sur mes recherches et qui m’ont aidé pour la rédaction de ce mémoire.
    
    En commençant par remercier tout d’abord Claudine et David pour m'avoir donné la chance de faire cette thèse.
    
    Merci Claudine, de m'avoir accueilli dans ton laboratoire, de ton aide précieuse et pour le temps consacré à enrichir ce mémoire.
    
    Merci David, j'ai aimé travailler avec toi, tu es passionné tout comme nos échanges, et souvent nous étions d'accord, mais nous employions des mots différents. Retranscrire l'intuition humaine vers un raisonnement logique n'a pas toujours était simple, alors quand il fallait l'adapter au métabolisme avec par endroit des pressentiments humains "la biologie est une science remplis d'exception \ldots" . On peut le dire, la problématique biologique nous a permis ensemble de relever ce challenge que l'on pensait résolu depuis longtemps par les logiciens. Grâce à nos échanges et notre ténacités on a permis modestement d'apporter notre contribution à la science.
    Encore une fois merci pour tout David.
    
    Je remercie Alexandra, pour ses moments passés à discuter lors de nos bouts de chemin en RER. Lors d'une thèse on rencontre inévitablement des moments difficiles et dans ces moments il a été agréables d'avoir quelqu'un avec qui discuter.
    
    Je pense également à Stéphane, je te remercie pour ces ballades à vélo, nos discussions techniques et ainsi que toutes les autres \ldots autant de bon moments passés ensemble.
    
    Je remercie Mark et Karine c'est toujours un plaisir de discuter avec vous et j'espère pouvoir échanger encore d'autres moments avec vous.
    
    Je te remercie Zoé, aux premiers abords on peut avoir peur, tu martyrise les plantes avec un ciseau et tu les empoisonnes avec du café. Bon, on n'a toujours pas de plante qui fait le café mais on découvre rapidement que tu es une personne gentille que l'on s'attache facilement.
    
    Forcément j'oublie pas "petit" David (mais grand par l'amitié que je te porte), travailleur et passionné, pendant longtemps tu étais un des derniers à partir du laboratoire. D'ailleurs c'était à de tels moments que l'on avait le plus souvent l'occasion de discuter. Merci pour tous ces moments partagés.
    
    Un grand merci à toute l'équipe du LABGeM, Aurélie pour sa bienveillance, Adrien pour nos nombreux échanges techniques, Alexandre avec qui on a partagé au début des problématiques similaires (nemometa, drools, java, maven et j'en passe \ldots) et à tous les autres.
    
    Je tiens à remercier également Claude Scarpelli pour sa bienveillance et nos diverses discussions abordées dans le RER.
    
    Un grand merci à Jean-Marc Aury pour m'avoir accueilli dans son équipe. Tu m'as permis d'acquérir de l'expérience sur les thématiques de l'assemblage de séquences.
    
    Je pense également à Arnaud, Frédérick et Stefan on a passé de très bons moments, le GenoFoot, les soirées foot mais pas seulement. Ne changez rien il est agréable d'être à vos côtés.
    
    Je remercie de façon plus large à toute la grande famille du GenoScope, je vous souhaite à tous que du bonheur.
    
    Enfin, je tenais à remercier mes parents, mon petit frère, ma petite sœur et tous mes proches qui m'ont accompagné, aidé, soutenu et encouragé tout au long de la réalisation de cette thèse. Et, plus que tout, merci à ma compagne, Lucie, pour son amour et son attention, toujours là, dans les hauts comme dans les bas, ce qui m'a permis de mener cette thèse avec succès.

%\vspace*{-1cm}
%\raggedleft \includegraphics[height=2cm]{img/merci.jpg}
\begin{textblock}{3}(14,24)
    \includegraphics[height=2cm]{img/merci.jpg}
\end{textblock}

\normalFont
