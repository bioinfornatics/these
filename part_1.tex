\chapter{De l'information à la connaissance}
Nous vivons une période, qui repoussent la limite des connaissances. La recherche est mondialisée… les outils d'analysent se démocratisent… de nombreuses plateformes d'application viennent enrichir ces données bruts… les super-calculateurs sont de plus en plus rapides, générant une quantité toujours plus importantes de nouvelles informations. Rajouter à ce constat, la vitesse quasi instantané d'échange et de communication des travaux scientifiques. Et nous obtenons une barrières entre information et connaissance de plus en plus flou.

En effet,  nous ne pouvons plus mener des recherches sur des vastes domaines comme "la Biologie", "les Mathématiques", "la Physique". Les connaissances à maitriser sont trop nombreuses. Les chercheurs, se spécialisent de plus en plus afin de maitriser un domaine toujours plus pointus. Mais malgré tous ces efforts, dans de nombreux domaines, il nous est impossible d'appréhender tous les travaux liés. Or ces connaissances, nous sont non seulement
importantes pour valider ou non les prédictions in-silico; mais il nous est également nécessaire d'élargir nos connaissances, hors de notre domaine d'expertise pour mieux le comprendre en retour.

Ce constat se vérifie en biologie, avec l'avènement des séquenceurs nouvelles génération. Le séquençage des organismes est devenu abordable et rapide. Ouvrant la voie à de très nombreux projets de recherche. Nous permettant d'avoir rapidement le code génétiques de tout organisme, également nommé séquence \gls{ADN}. Cette séquences peut faire quelque centaines de millier à plusieurs centaines de million de paire de bases nucléiques. Des outils bio-informatiques analysent ces séquences afin de prédire les régions géniques et leur fonction dans l'organisme. Le nombre de gène peut aller de quelques centaines à plusieurs dizaine de millier selon les organismes. Par conséquent l'expertise humaine d'un génome et un défi en soi. Ces recherches se sont intensifiées et complexifiées, comme l'étude de 100 000 génomes d'organismes pathogènes \cite{100kfoodborne}, ou encore sur les eco-systèmes poly-microbiens présent sur l'Homme \cite{hmp}. Pour se faire des outils bio-informatiques ont automatisé le traitement de l'information issue des séquenceurs afin de traiter un nombre d'organisme toujours plus grand. Ces outils, alimentés continuellement en nouveaux génomes, ont amplifié le déluge d'information. Dans le domaine de l'annotation fonctionnelle, moins d'un pour cent des données ont put être vérifiés (au regards des statistiques publiés par uniprot et swissprot \parencites{uniprot_stat}{expasy_stat} ). Ce fossé entre information de confiance et prédiction s'accélère.

Or les prédicteurs automatiques de fonctions génique ne sont pas fiables. En effet, 30\% des annotations fonctionnelles seraient incorrectes, voire 80\% dans certaines familles de protéines \parencites{devos2001intrinsic}{schnoes2009annotation}. Ces séquences incorrectement annotées sont ensuite propagées dans les bases de connaissances. Une des méthodes d'assignation de fonction à un gène, consiste de rechercher une séquence proches, dont la fonction est connue. Supposant qu'ils partagent une même fonction. Ainsi on inférera l'annotation à la séquence de fonction inconnue car elle est proche de la première. Cette pratique tant a amplifié l'inexactitude des fonctions géniques. En effet, les séquences mal annotées se retrouvent parmi les autres et sont donc potentiellement ré-utilisées \unsure{est-ce utile de remettre une couche?} afin d'inférer une fonction un autre gène considérait proche. Détériorant un peu plus la qualité de ces base de données.

L'objectif de l'annotation des fonctions géniques et de fournir un catalogue des capacités dont est pourvu un organisme, et donc a posteriori de mieux comprendre le vivant. Mais lorsque le processus d'annotation, produit et amplifie l'assignation de mauvaise fonctions aux gènes. Cela a pour effet de ne plus pouvoir utiliser ces prédictions sans prendre un risque. De plus, ce catalogue de fonctions géniques est utilisé par la suite dans de nombreux domaines, comme l'étude des voies métaboliques, la biologie des systèmes, la classification des gènes essentiels et autres. Cette problématique impacte notre compréhension du vivant et notre capacité à l'étudier, remettant en cause tous le processus d'annotation des gènes utilisé jusqu'alors.

Face à cette problématique des approches variées ont été développées. On distingue les systèmes d'annotations automatiques à base de règle, reprenant le raisonnement appliqué par les bio-curateurs, comme le projet HAMAP \cite{lima2009hamap}. Ces règles sont généralement basé sur la séquence génomique et la taxonomie de l'organisme. Elle peuvent être crée par un bio-curateur ou par des outils d'apprentissage \cite{uniprot2011ongoing}.

On retrouve également les systèmes de reconstruction des voies métaboliques \cite{karpe2011pathway}, utilisant les prédictions de fonction génique, spécifiques d'un organisme afin de proposé des voies métaboliques décrites dans d'autres organismes. Ces prédictions forment un graphe de connaissance, spécifique de l'organisme. Ainsi, des fonctions marquées comme manquante à la réalisation des voies métaboliques sont suggérées aux bio-curateurs.

Les méthodes à bases de règles automatise l'annotation fonctionnelle par l'utilisation de règle lié à la biologie de l'organisme et non plus uniquement par une prédiction in-silico.  Les méthodes utilisant la représentations des connaissances biologiques permettent de contextualiser les prédictions et de suggérer des annotations fonctionnelles ne pouvant être détecté in-silico. Toutefois le travail du bio-curateur bien que nécessaire, il est considéré comme trop long, laborieux et source d'erreur.

%L'équipe HELIX dirigé par Alain Viari (INRIA) à développé un prototype de vérification de la cohérence globale de l'annotation (HERBS). Reprenant les systèmes à base de règle et la représentation des connaissances. Ainsi les voies métaboliques peuvent être rattaché à des caractères phénotypique. Par exemple, la pousse d'un organisme sur un milieu avec une seule source de carbone, un antibiotique \ldots peuvent être relié au graphe de connaissance et ainsi comparé les prédictions par rapports aux attentes issues des données expérimentales. 


%Cet discipline consiste à caractériser la fonction des gènes, afin de lister les fonctionnalités dont est pourvu un organisme. En amont de ce travail les outils bio-informatiques émettent des prédictions sur les fonctions des gènes identifiés. Toutefois seulement une partie de ces prédictions sont vérifiées expérimentalement ou par un curateur. Or de nombreuses études montre les limites de l'annotation fonctionnelle automatiques. 



\section{Représentation des connaissances}
\subsection{Représentation des voies métaboliques}
\subsubsection{KEGG}
\subsubsection{Reactome}
\subsubsection{Unipathway}
\subsubsection{Genome properties}
\subsection{Représentation des résultats expérimentaux}
néant

\section{Nature de l'information biologique}
\subsection{Observation}
\subsection{Prédiction}
\subsection{Expectation}

\section{De l'annotation géniques aux métabolismes}
\subsection{Annotations fonctionnelles des génomes}
\subsection{Reconstruction de réseaux et modèles métaboliques}
\section{Les analyses bioinformatiques}
\section{Les expérimentations en laboratoire}

\section{Des résultats à l'interprétation automatique}
\subsection{La biologie un monde d'incertitude}
\subsection{Pluralité de l'information}
diversité des résultats contradiction
\subsection{La logique classique}
\subsection{La logique multi-valuée}
belnap and co
\subsection{HERBS}
