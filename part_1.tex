\chapter{Introduction}
\section{De l'information à la connaissance}
Nous vivons une période, qui repoussent la limite des connaissances. Les chercheurs du monde entier, travail sur des sujets, tous plus passionnant les uns que les autres autres. Les super-calculateurs, génèrent sans cesses des données, pour restituer d'autant plus de nouvelles informations. Rajouter à ce constat, la vitesse quasi instantané d'échange et de communication \deleted{des travaux scientifiques}. Et nous obtenons une barrières entre information et connaissance de plus en plus flou.

Ce constat se vérifie en biologie, avec l'avènement des séquenceurs nouvelles génération. Le séquençage des organismes est devenu abordable et rapide. Ouvrant la voie à de très nombreux projets de séquençage. Ces recherches se sont intensifiées et complexifiées, comme l'étude de 100 000 génomes d'organismes pathogènes \cite{100kfoodborne}, ou encore sur les eco-systèmes microbiens présent sur l'Homme \cite{hmp}. Pour se faire des outils bio-informatiques ont automatisé le traitement de l'information issue de séquenceur afin de traiter un nombre d'organisme toujours plus grand. Ces outils, alimentés continuellement en nouveaux génomes, ont amplifié le déluge d'information. D'une proportion telle, que seulement moins d'un pour cent des données ont put être vérifiés (au regards des statistiques publiés par uniprot et swissprot \parencites{uniprot_stat}{expasy_stat} ). Ce fossé entre information de confiance et prédiction s'accélère.

Or la prédictions automatiques des fonctions géniques n'est pas fiable. En effet, 30\% des annotations seraient incorrectes, voire 80\% dans certaines familles de protéines \parencites{devos2001intrinsic}{schnoes2009annotation}. Ces annotations incorrectes sont ensuite propagées dans les bases de données puis ré-utilisées afin d'inférer la fonction du gène sur un autre  gène considérait proche. Détériorant un peu plus la qualité de ces base de données.
L'objectif de l'annotation des fonctions géniques et de fournir un catalogue des capacités dont est pourvu un organisme, et donc a posteriori de mieux comprendre le vivant. Mais lorsque le processus d'annotation produit et amplifie l'assignation de mauvaise fonctions aux gènes. Cela a pour effet de ne plus pouvoir utiliser ces prédictions sans prendre un risque. De plus, ce catalogue de fonctions géniques est utilisé par la suite dans de nombreux domaines, comme l'étude des voies métaboliques, la biologie des systèmes, la classification des gènes essentiels et autres. Cette problématique impacte notre compréhension du vivant et notre capacité à l'étudier, remettant en cause tous le processus d'annotation des gènes utilisé jusqu'alors. 


   Bien que nombreux efforts de curation des prédictions fonctionnelles sont réalisées. pour transformer l'information en connaissance. Mais la quantité d'information généré dépasse largement nos capacités. Face à cet problématique nous avons cherché a apporter une solution dans un domaine plus restreint, qui est l'annotation fonctionnelles des gènes prokaryotes.

Cet discipline consiste à caractériser la fonction des gènes, afin de lister les fonctionnalités dont est pourvu un organisme. En amont de ce travail les outils bio-informatiques émettent des prédictions sur les fonctions des gènes identifiés. Toutefois seulement une partie de ces prédictions sont vérifiées expérimentalement ou par un curateur. Or de nombreuses études montre les limites de l'annotation fonctionnelle automatiques. 




%\definition{Savoir}{}
%\chapter{Les données biologiques}
%\section{Representation du métabolisme}
%\subsection{BioCyc et MetaCyc}
%\subsection{Unipathway}
%\subsection{KEGG}
%\section{De l'annotation geniques aux métabolismes}
%\subsection{Annotations fonctionnelles des génomes}
%\subsection{Reconstruction de réseaux et modèles métaboliques}