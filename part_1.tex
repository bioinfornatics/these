\begin{refsection}
	\chapter{Contexte biologique et méthodologique }

    \section{Le métabolisme et sa représentation informatique}
    \subsection{Généralités sur le métabolisme}
    \subsection{Les acteurs}
    \subsection{Représentation en graphe}
    \subsection{Ressources sur les voies métaboliques}
    \subsubsection{KEGG}
    \subsubsection{Reactome}
    \subsubsection{Unipathway}
    \subsubsection{Genome properties}
    
    \section{Des génomes aux réseaux métaboliques}
    \subsection{Annotation fonctionnelle}\label{subsect:annotation}
    \subsection{Reconstruction des réseaux métaboliques}
    \subsection{Modèles métaboliques}
    \subsection{Les données expérimentales}
    \subsubsection{Élucidation des voies métaboliques}
    \subsubsection{Phénotypes de croissances}
    
    \section{Raisonnement logique dans le processus de curation}
    \subsection{Lacunes et incertitudes dans nos connaissances}
    \subsubsection{Les trous dans les connaissances et les enzymes orphelines}
    \subsubsection{Limites de l’annotation fonctionnelle et rôle de la curation}
    \subsection{Logique et raisonnement}
    \subsubsection{Les différentes logiques}
    \paragraph{Logique booléenne}
    \paragraph{Logique multi-valuée}
    \subsubsection{Inférence d’information}
    \paragraph{Représenation des connaissances/ontologies}
    \paragraph{Chainage avant et arrière} %backward/forward
    \paragraph{Règles et système expert}
    
    \section{Méthodes existantes}
    \subsection{HAMAP et UniRule}
    \subsection{Genome properties}
    \subsection{IMG terms}
    \subsection{HERBS}
    
    \begingroup
        \setlength\bibitemsep{1em}
        \printbibliography[segment=\therefsegment,heading=subbibliography]
    \endgroup
\end{refsection}