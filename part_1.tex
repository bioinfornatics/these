\chapter{Introduction}
\section{De l'information à la connaissance}
Nous vivons une période, qui repoussent la limite des connaissances. Les chercheurs du monde entier, travail sur des sujets, tous plus passionnant les uns que les autres autres. Les super-calculateurs, génèrent sans cesses des données, pour restituer d'autant plus de nouvelles informations. Rajouter à ce constat, la vitesse quasi instantané d'échange et de communication \deleted{des travaux scientifiques}. Et nous obtenons une barrières entre information et connaissance de plus en plus flou.

Ce constat se vérifie en biologie, avec l'avènement des séquenceurs nouvelles génération facilitant le séquençage rapide des organismes. De plus, les outils bio-informatiques, alimenté continuellement en nouveaux génomes, ont amplifié le déluge d'information. D'une proportion telle, que la quantité d'information vérifié représente moins d'un pour cent des données produites de façon automatique (au regards statistiques publiés par uniprot et swissprot \parencites{uniprot_stat}{expasy_stat} ). Bien que nombreux efforts de curation des prédictions fonctionnelles sont réalisé. pour transformer l'information en connaissance. Mais la quantité d'information généré dépasse largement nos capacités. Face à cet problématique nous avons cherché a apporter une solution dans un domaine plus restreint, qui est l'annotation fonctionnelles des gènes prokaryotes.

Cet discipline consiste à caractériser la fonction des gènes, afin de lister les fonctionnalités dont est pourvu un organisme. En amont de ce travail les outils bio-informatiques émettent des prédictions sur les fonctions des gènes identifiés. Toutefois seulement une partie de ces prédictions sont vérifiées expérimentalement ou par un curateur. Or de nombreuses études montre les limites de l'annotation fonctionnelle automatiques. En effet, 30\% des annotations seraient incorrecte, voire 80\% dans certaines familles de protéines \parencites{devos2001intrinsic}{schnoes2009annotation}.




%\definition{Savoir}{}
%\chapter{Les données biologiques}
%\section{Representation du métabolisme}
%\subsection{BioCyc et MetaCyc}
%\subsection{Unipathway}
%\subsection{KEGG}
%\section{De l'annotation geniques aux métabolismes}
%\subsection{Annotations fonctionnelles des génomes}
%\subsection{Reconstruction de réseaux et modèles métaboliques}